% this is the TeX template used for the hardware specifications. the fields
% labelled with <++> must be completed.

\documentclass{scrreprt}

% allow images
\usepackage{graphicx}
% allow chinese
\usepackage{CJK}
% lists
\usepackage{enumitem}
% hyperlinked TOC (note the CJKbookmarks boolean)
\usepackage[CJKbookmarks = true]{hyperref}
% allow figures and tables to hold position
\usepackage{float}
% pretty tables
\usepackage{booktabs}
\usepackage[table]{xcolor}
% symbols
\usepackage{gensymb}
% format headers and footers
\usepackage[automark,headsepline,footsepline,plainfootsepline]{scrlayer-scrpage}
% allow for usage of \Blinddocument to test formatting
\usepackage{mwe}

% options
% <++> \graphicspath{{./figures/}}
\definecolor{light-gray}{gray}{0.95}

% define macros
\newcommand{\pchapter}[1]{
	\begingroup\let\clearpage\relax
	\newpage
	\begin{figure}[H]
		\includegraphics[width=0.25\textwidth]{logo.jpeg}
	\end{figure}
	\chapter{#1}
	\endgroup
}
\newcommand{\modelno}{
	\texttt{<++>}
}
\newcommand{\upstream}{
	\small\texttt{https://github.com/diracs-delta/fruition-specs/%
		      tree/master/<++>}
}
\newcommand{\x}{
	$\times$
}
\newenvironment{ptable}[2][def]
{
	\begin{table}[H]
	\centering
	\rowcolors{1}{white}{light-gray}
	\caption{#1}
	\begin{tabular}{#2}
	\toprule
}
{
	\bottomrule
	\end{tabular}
	\end{table}
}

% define header, title, date
\lohead{\includegraphics[width=\marginparwidth]{logo.jpeg}}
\title{
	\begin{figure}[H]
		\centering\includegraphics[width=0.5\textwidth]{logo.jpeg}
	\end{figure}
	\vspace{1cm}
	\flushright
%	<++>
	\Huge{DATA SHEET TEMPLATE}\\
	\vspace{2cm}
	\huge{Model No. \modelno}\\
	\vspace{2cm}
	\LARGE{Prepared by David Qiu \\ on behalf of FRUITION CO., LTD.}
}
\date{
%	<++>
	Last revision: January 11th, 2019\\
	\vspace{0.5cm}
	Full revision history and latest data sheets are available at\\
	\vspace{0.25cm}
	\upstream.
}
%%%%%%%%%%%%%%%%%%%%%%%%%%%%%%%%%%%%%%%%%%%%%%%%%%%%%%%%%%%%%%%%%%%%%%%%%%%%%%%%
\begin{document}
\begin{CJK*}{UTF8}{gbsn}
\maketitle
\tableofcontents

\pchapter{Template Overview}
\section{Custom Macros}
\begin{itemize}
\item \verb|\pchapter{#1}| -- Prints a chapter header (\verb|#1|) with a
left-justified image (\verb|logo.jpeg| by default) above it. This chapter is
formatted using \verb|\pchapter|; the next one isn't.

\item \verb|\upstream| -- Prints the upstream URL formatted in the Typewriter
font family.

\item \verb|\modelno| -- Prints model number formatted in the Typewriter font
family.
\end{itemize}

\section{Custom Environments}
\begin{itemize}
\item \verb|ptable[1]{2}| -- Used to format tables using \verb|booktabs| and
\verb|tabular| captioned above by \verb|#1|. \verb|#2| defines the alignment of
the text in each column, with one letter per column, read left-to-right. See
\verb|tabular| documentation for more information.

	\begin{itemize}
	\item Example:

	\begin{verbatim}
	\begin{ptable}[绝对最大额定值]{ccccc}
	参数     & 符号 & 条件& 额定值          & 单位      \\
	\midrule
	供电电压 & VDD  & --  & 0 - 3.8         & V         \\
	静电防护 & ESD  & HBM & 2               & kV        \\
	储存温度 & TSTG & --  & $-40$ -- $+125$ & \degree C \\
	\end{ptable}
	\end{verbatim}

	\begin{ptable}[绝对最大额定值]{ccccc}
	参数     & 符号 & 条件& 额定值          & 单位      \\
	\midrule
	供电电压 & VDD  & --  & 0 - 3.8         & V         \\
	静电防护 & ESD  & HBM & 2               & kV        \\
	储存温度 & TSTG & --  & $-40$ -- $+125$ & \degree C \\
	\end{ptable}
	\end{itemize}
\end{itemize}


\Blinddocument

\end{CJK*}
\end{document}
%%%%%%%%%%%%%%%%%%%%%%%%%%%%%%%%%%%%%%%%%%%%%%%%%%%%%%%%%%%%%%%%%%%%%%%%%%%%%%%%
