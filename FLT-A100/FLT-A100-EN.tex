\documentclass{scrreprt}

% allow images
\usepackage{graphicx}
% allow chinese
\usepackage{CJK}
% lists
\usepackage{enumitem}
% hyperlinked TOC (note the CJKbookmarks boolean)
\usepackage[CJKbookmarks = true]{hyperref}
% allow figures and tables to hold position
\usepackage{float}
% pretty tables
\usepackage{booktabs}
\usepackage[table]{xcolor}
% symbols
\usepackage{gensymb}
% format headers and footers
\usepackage[automark,headsepline,footsepline,plainfootsepline]{scrlayer-scrpage}
% allow for usage of \Blinddocument to test formatting
\usepackage{mwe}

% options
\graphicspath{{./figures/} {../}}
\definecolor{light-gray}{gray}{0.95}

% define macros
\newcommand{\pchapter}[1]{
	\begingroup\let\clearpage\relax
	\newpage
	\begin{figure}[H]
		\includegraphics[width=0.25\textwidth]{logo.jpeg}
	\end{figure}
	\chapter{#1}
	\endgroup
}
\newcommand{\modelno}{%
	\texttt{FLT-A100}
}
\newcommand{\upstream}{
	\small\texttt{https://github.com/diracs-delta/fruition-specs/%
		      tree/master/FLT-A100-EN}
}
\newcommand{\x}{
	$\times$
}
\newenvironment{ptable}[2][def]
{
	\begin{table}[H]
	\centering
	\rowcolors{1}{white}{light-gray}
	\caption{#1}
	\begin{tabular}{#2}
	\toprule
}
{
	\bottomrule
	\end{tabular}
	\end{table}
}

% define header, title, date
\lohead{\includegraphics[width=\marginparwidth]{logo.jpeg}}
\title{
	\begin{figure}[H]
		\centering\includegraphics[width=0.5\textwidth]{logo.jpeg}
	\end{figure}
	\vspace{1cm}
	\flushright
	\Huge{IMU MODULE}\\
	\Huge{HARDWARE SPECIFICATION}\\
	\vspace{2cm}
	\huge{Model No. \modelno}\\
	\vspace{2cm}
	\LARGE{Prepared by David Qiu \\ on behalf of FRUITION CO., LTD.}
}
\date{
	Last revision: January 11th, 2019\\
	\vspace{0.5cm}
	Full revision history and latest data sheets are available at\\
	\vspace{0.25cm}
	\upstream.
}
%%%%%%%%%%%%%%%%%%%%%%%%%%%%%%%%%%%%%%%%%%%%%%%%%%%%%%%%%%%%%%%%%%%%%%%%%%%%%%%%
\begin{document}
\begin{CJK*}{UTF8}{gbsn}
\maketitle
\tableofcontents

\pchapter{Product Overview}
\section{General Description}
The \modelno is Fruition's latest inertial measurement unit (IMU) module, an
economical solution to any hardware application in need of gyroscopic or
accelerometric measurements.

The \modelno IMU module is a tri-axis accelerometer and gyroscope, integrated
with an energy-efficient microprocessor suitable for low power applications. The
module firmware also contains an advanced signal processing algorithm aimed at
noise reduction. It also contains a sensor fusion algorithm that outputs its
three principal axes, i.e.\ its roll, pitch, and yaw. The module also supports
UART communication through its stamp half-hole patch interface.

Our module is especially suitable towards robotics industries necessitating
modules that are space-efficient. The \modelno IMU module is only
15.2\x17.8 mm in size, making it perfect for such applications.

\section{Features}
\begin{itemize}
\item Small size that is competitive with existing MEMS sensors.

\item Integrated high-precision six-axis gyroscope.

\item UART interface with high baud rate and I/O frequency.

\item Accurate sensor fusion algorithm that calculates the roll, pitch, and yaw.

\item High stability against thermal and vibrational fluctuation.

\item Low power consumption.
\end{itemize}

\pchapter{Legal}
\section{Safety Information}
This device is sensitive to electrostatic discharge (ESD). Though the module
comes with standard ESD circuit protection, the components may still malfunction
if ESD occurs. Standard ESD precautions should be taken when handling this
device.

\section{Liability}
\textbf{SHENZHEN FRUITION CO., LTD.} shall not be liable, under any
circumstances, for any special, indirect, incidental, consequential, or
contingent damages for any reason, whether or not the buyer has been advised of
the possibility of such damage.

\end{CJK*}
\end{document}
%%%%%%%%%%%%%%%%%%%%%%%%%%%%%%%%%%%%%%%%%%%%%%%%%%%%%%%%%%%%%%%%%%%%%%%%%%%%%%%%
